\documentclass{article}

\usepackage{multicol}
\usepackage{lscape}
\usepackage{enumitem}
\usepackage[letterpaper, landscape, margin=0.75in]{geometry}

\usepackage{helvet}
\renewcommand{\familydefault}{\sfdefault}

\pagenumbering{gobble}
\setlist[itemize]{leftmargin=*,label={},itemsep=0ex}
\setlist[description]{style=unboxed,leftmargin=*,itemsep=0ex}

\usepackage{fancyhdr}
\pagestyle{fancy}
\fancyhf{}
\lhead{Feedback Skills For Software Developers Cheatsheet}
\rhead{prealpha-0.0.2}

\begin{document}

\begin{multicols}{3}

	\section*{Receving Positive Feedback}

        A compliment is a gift to be accepted. It is not a bomb to be defused, nor a volleyball to be returned.

        \begin{description}
            \item[Thank you.]
            \item[Thanks. I worked hard on that.]
            \item[I'm glad you liked it.]
            \item[Thanks. I appreciate that you noticed.]
            \item[Thanks. I feel good about it too.]
        \end{description}

	\section*{Giving Helpful Positive Feedback}

        \begin{description}
            \item[Refrain from the false complement.] \emph{I found that two thousand line method very easy to understand.}
            \item[Avoid the backhanded compliment.] \emph{Your code is so good that it doesn't need unit tests.}
            \item[Resist complementing future behavior.] \emph{You're really responsive and helpful. Could you please help me solve this problem.}
            \item[Complement behavior that has occurred.] \emph{It was really responsive of you to help me with that problem on short notice.}
            \item[Be specific.] \emph{I find your use of short, single responsibility methods makes the code very readable.}
            \item[Use shaping.] Reward gradual approximations of the goal. \emph{Well done. Your pull request increased overall test coverage from 27\% to 30\%.}
        \end{description}

    \columnbreak 

	\section*{Receving Corrective Feedback}

        \begin{description}
            \item[Relax.] Open your mouth slightly. Soften your eyes. Open your hands. Relax your belly.
            \item[Avoid retaliation.] \emph {Well, it isn't like it's easy to read your forced use of recursion.}
            \item[Hold back.] This is sometimes called `elective blindness' or `elective deafness'. In other words, ignore the nasty face or sharp tone.
            \item[Consider your safety.] If violence is a risk, it is often not worth being assertive.
            \item[Don't demand perfection.] Understand that most people simply aren't that skilled at giving feedback. Probe for value.
            \item[Validate their perceptions.] \emph{I can see how you might find this use of recursion awkward. A \texttt{while} loop might have been more natural.}
            \item[Validate their emotions.] \emph{This is important to you.} \emph{You're really quite concerned about this.}
            \item[Agree in part.] \emph{You're right. Opinionated, automated formatting does have its downsides.}
            \item[Listen and wait.] Let the critic voice their concern completely. Take the criticism \emph{in} without taking it \emph{on}.
            \item[Narrow and specify.] \emph{What's one downside of opinionated, automated formatting that comes to mind?}
            \item[Ask for clarification.] \emph{You mentioned a preference for 160 character lines. How important is that to you?}
            \item[Explain without offering excuses.] \emph{I used recursion here because I find it easier to reason about immutable code.}
            \item[Don't try to change their mind.] \emph{Agree to disagree. It's rare for a resolution to require full agreement.}
            \item[Thank the critic.] We need feedback. Thanking them reminds us of this.
            \item[Respond to the style.] \emph{That was really helpful, specific feedback - thank you for speaking one-on-one.}
            \item[Ask for time.] \emph{Thank you for bringing up my persistence on problems. How about we meet next week, once I've had some time to consider what you said.}
            \item[Acknowledge.] \emph{Yes. I was half an hour late for work today. My car battery died.}
            \item[Cloud.] Use for statements that have a grain of truth but are intended mostly as insults.
              \begin{itemize}
                 \item Agree in part. \emph{Yes. I do work more than 40 hours a week.}
                 \item Agree in probability. \emph{It could be that I work too much.}
                 \item Agree in principle. \emph{You're right. If I work too much, I will burn out.}
              \end{itemize}
            \item[Probe.] \emph{What is it that bothers you about [important part of feedback]?}
         \end{description}
 
	\section*{Giving Corrective Feedback}

        \begin{description}
            \item[Choose your timing.]
            \item[Watch the ratio.]
            \item[Think before talking.]
            \item[Talk one-to-one.]
            \item[Frame the issue.]
            \item[Be precise.]
            \item[Include the positive in the message.]
            \item[Give information, not advice.]
            \item[Don't emote.]
        \end{description}

\end{multicols}

\begin{itemize}
    \item Paterson, R. J. (2000). The assertiveness workbook: How to express your ideas and stand up for yourself at work and in relationships. Oakland, CA: New Harbinger Publications.
    \item McKay, M., Davis, M., \& Fanning, P. (2009). Messages: The Communication Skills Book. Oakland, CA: New Harbinger Publications.
\end{itemize}


\end{document}
