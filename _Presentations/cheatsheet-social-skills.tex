\documentclass{article}

\usepackage{multicol}
\usepackage{lscape}
\usepackage{enumitem}
\usepackage[letterpaper, landscape, margin=0.7in]{geometry}

% See https://tex.stackexchange.com/q/131742/137042
\usepackage[most]{tcolorbox}

\usepackage{helvet}
\renewcommand{\familydefault}{\sfdefault}

\usepackage[tiny]{titlesec}

\pagenumbering{gobble}
\setlist[itemize]{leftmargin=*,label={},itemsep=0ex}
\setlist[description]{style=unboxed,leftmargin=*,itemsep=0ex}

\usepackage{fancyhdr}
\pagestyle{fancy}
\fancyhf{}
\lhead{Feedback Skills for Software Developers Cheatsheet}
\rhead{prealpha-0.0.3}

% https://www.overleaf.com/learn/latex/Using_colours_in_LaTeX
\usepackage{xcolor}

\definecolor{salmon}{rgb}{0.98,0.50,0.45}

\colorlet{rainbow1}{red}
\colorlet{rainbow2}{salmon}
\colorlet{rainbow3}{yellow}
\colorlet{rainbow4}{green}
\colorlet{rainbow5}{blue}
\colorlet{rainbow6}{purple}
\colorlet{rainbow7}{violet}

\begin{document}
\begin{tiny}

\begin{multicols}{3}

    \begin{tcolorbox}[
            enhanced,
            coltitle=black,
            colbacktitle={white},
            colback={white},
            title=Receving Positive Feedback,
            colframe=rainbow1
        ]

        A compliment is a gift to be accepted. It is not a bomb to be defused, nor a volleyball to be returned.

        \begin{description}
            \item[Thank you.]
            \item[Thanks. I worked hard on that.]
            \item[I'm glad you liked it.]
            \item[Thanks. I appreciate that you noticed.]
            \item[Thanks. I feel good about it too.]
        \end{description}

    \end{tcolorbox}

    \begin{tcolorbox}[
            enhanced,
            coltitle=black,
            colbacktitle={white},
            colback={white},
            title=Giving Helpful Positive Feedback,
            colframe=rainbow2
        ]

        \textbf{Barbara Fredrickson}
        \begin{itemize}
            \item People are like an aperature.
            \item Negative feedback has more weight for the reciever.
            \item Many people have a secret sense of incompetence.
            \item Desensitize people by using the 3:1 ratio.
            \item This takes a concerted effort!
            \item Focus within shooting distance.
        \end{itemize}

        \begin{description}
            \item[Refrain from the false complement.] \emph{Wow. I love how you put so many responsibilities into that one class.}
            \item[Avoid the backhanded compliment.] \emph{Your test coverage is so much better, not nearly so lazy as before.}
            \item[Resist complementing future behavior.] \emph{You're really responsive and helpful. Could you please help me solve this problem.}
            \item[Complement behavior that has occurred.] \emph{It was really responsive of you to help me with that problem on short notice.}
            \item[Be specific.] \emph{I find your use of short, single responsibility methods makes the code very readable.}
            \item[Use shaping.] Reward gradual approximations of the goal. \emph{Well done. Your pull request increased overall test coverage from 27\% to 30\%.}
        \end{description}

    \end{tcolorbox}

    \begin{tcolorbox}[
            enhanced,
            coltitle=black,
            colbacktitle={white},
            colback={white},
            title=Receiving Corrective Feedback,
            colframe=rainbow3
        ]

        \begin{description}
            \item[Relax.] Open your mouth slightly. Soften your eyes. Open your hands. Relax your belly.
            \item[Avoid retaliation.] \emph {Well, it isn't like it's easy to read your forced use of recursion.}
            \item[Hold back.] This is sometimes called `elective blindness' or `elective deafness'. In other words, ignore the nasty face or sharp tone.
            \item[Consider your safety.] If violence is a risk, it is often not worth being assertive.
            \item[Don't demand perfection.] Understand that most people simply aren't that skilled at giving feedback. Probe for value.
            \item[Validate their perceptions.] \emph{I can see how you might find this use of recursion awkward. A \texttt{while} loop might have been more natural.}
            \item[Validate their emotions.] \emph{This is important to you.} \emph{You're really quite concerned about this.}
            \item[Agree in part.] \emph{You're right. Opinionated, automated formatting does have its downsides.}
            \item[Listen and wait.] Let the critic voice their concern completely. Take the criticism \emph{in} without taking it \emph{on}.
            \item[Narrow and specify.] \emph{What's one downside of opinionated, automated formatting that comes to mind?}
            \item[Ask for clarification.] \emph{You mentioned a preference for 160 character lines. How important is that to you?}
            \item[Explain without offering excuses.] \emph{I used recursion here because I find it easier to reason about immutable code.}
            \item[Don't try to change their mind.] \emph{Agree to disagree. It's rare for a resolution to require full agreement.}
            \item[Thank the critic.] We need feedback. Thanking them reminds us of this.
            \item[Respond to the style.] \emph{That was really helpful, specific feedback - thank you for speaking one-on-one.}
            \item[Ask for time.] \emph{Thank you for bringing up my persistence on problems. How about we meet next week, once I've had some time to consider what you said.}
            \item[Acknowledge.] \emph{Yes. I was half an hour late for work today. My car battery died.}
            \item[Cloud.] Use for statements that have a grain of truth but are intended mostly as insults.
              \begin{itemize}
                 \item Agree in part. \emph{Yes. I do work more than 40 hours a week.}
                 \item Agree in probability. \emph{It could be that I work too much.}
                 \item Agree in principle. \emph{You're right. If I work too much, I will burn out.}
              \end{itemize}
            \item[Probe.] \emph{What is it that bothers you about [important part of feedback]?}
         \end{description}

    \end{tcolorbox}

    \begin{tcolorbox}[
            enhanced,
            coltitle=black,
            colbacktitle={white},
            colback={white},
            title=Giving Helpful Corrective Feedback,
            colframe=rainbow4
        ]

        \begin{description}
            \item[Choose your timing.] Your feedback is more likely to help if the recipient can focus on your message.
                \begin{itemize}
                    \item Worse: after a long, hard work day.
                    \item Better: after lunch.
                \end{itemize}
            \item[Watch the ratio.] Pointing out what people are doing right is more powerful than pointing out what they are doing wrong. What do we want them to repeat?
            \item[Think before talking.] It's easy to get off track. What do want you to say? How do you want to say it?
                \begin{itemize}
                    \item Worse: \emph{So, um, I wanted to talk about the code, you know, naming things is hard, and, do you know that quote...}                    
                    \item Better: \emph{Our code base uses whole words for variable names, and your pull request uses mostly single-character variable names.}
                \end{itemize}
            \item[Talk one-to-one.] This becomes less important as your feedback skills become more supportive and assertive.
                \begin{itemize}
                    \item Worse: In front of the whole team over zoom: \emph{Shaun, your pull request was full of unused private fields}.
                    \item Better: Via a private message over Slack: \emph{Shaun, could I meet with you over Zoom for a moment?}
                \end{itemize}
            \item[Frame the issue.] This is important if what you say is likely to be threatening.
                \begin{itemize}
                   \item Worse: \emph{I would like to talk about how long it took you to complete your last task... }
                   \item Better: \emph{Your job here is secure. I would lik eto talk about how long it took... }
                \end{itemize}
            \item[Be precise.] People need to know exactly what to change.
                \begin{itemize}
                   \item Worse: \emph{I didn't like your code} 
                   \item Better:\emph{Some files have different names than the classes they contained.} 
                \end{itemize}
            \item[Include the positive in the message.] This opens the aperature - at least a bit.
                \begin{itemize}
                    \item E.g: \emph{Here is what I think you did really well... here is what I thought could have been better.}
                    \item Q: How does this differ from the backhanded complement?
                \end{itemize}
            \item[Give information, not advice.] Where possible, let the other person decide what to do.
                \begin{itemize}
                    \item Worse. \emph{Put up your fly.} \emph{Convert the recursion to a while loop.}
                    \item Better \emph{Your fly is down.} \emph{Many devs on our team find it easier to reason about loops than about recursion.}
                \end{itemize}
            \item[Don't emote.] If you're passionate about the topic, say so, but avoid acting out your passion.
                \begin{itemize}
                    \item Worse. \emph{It pisses me of about this stupid use of recusion...}
                    \item Better \emph{I feel passionate about using loops instead of recursion, because...}
                \end{itemize}
        \end{description}

    \end{tcolorbox}

\end{multicols}

\begin{itemize}
    \item Paterson, R. J. (2000). The assertiveness workbook: How to express your ideas and stand up for yourself at work and in relationships. Oakland, CA: New Harbinger Publications.
    \item McKay, M., Davis, M., \& Fanning, P. (2009). Messages: The Communication Skills Book. Oakland, CA: New Harbinger Publications.
    \item Fredrickson, B. L. (2009). Positivity: Top-notch research reveals the 3-to-1 ratio that will change your life. New York: Three Rivers Press/Crown Publishers.
\end{itemize}

\end{tiny}
\end{document}
