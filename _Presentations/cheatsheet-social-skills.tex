% Build and open with PowerShell.
% kill -Name FoxitReader; pdflatex .\cheatsheet-social-skills.tex; ii .\cheatsheet-social-skills.pdf;

\documentclass{article}

\usepackage{multicol}
\usepackage{lscape}
\usepackage{enumitem}
\usepackage[letterpaper, landscape, margin=0.5in]{geometry}

\pagenumbering{gobble}
\setlist[itemize]{leftmargin=*}
\setlist[description]{style=unboxed,leftmargin=*}

\begin{document}

% TODO Use a command that's more appropriate than \section is.
\section*{Feedback Skills Cheatsheet}

\begin{multicols}{3}

	\section*{Receving Positive Feedback}

        A compliment is a gift to be accepted. It is not a bomb to be defused, nor a volleyball to be returned.

        \begin{description}
            \item[Thank you.]
            \item[Thanks. I worked hard on that.]
            \item[I'm glad you liked it.]
            \item[Thanks. I appreciate that you noticed.]
            \item[Thanks. I feel good about it too.]
        \end{description}

	\section*{Giving Positive Feedback}

        \begin{description}
            \item[Refrain from the false complement.]
            \item[Avoid the backhanded compliment.]
            \item[Complement behavior that has occurred.]
            \item[Be specific.]
            \item[Use shaping.]
        \end{description}

	\section*{Receving Corrective Feedback}

        \begin{description}
            \item[Relax.]
            \item[Avoid retaliation.]
            \item[Hold back.]
            \item[Consider your safety.]
            \item[Don't demand perfection.]
            \item[Validate their perceptions.]
            \item[Validate their emotions.]
            \item[Agree in part.]
            \item[Listen and wait.]
            \item[Narrow and specify. ]
            \item[Ask for clarification.]
            \item[Explain without offering excuses.]
            \item[Don't try to change their mind. ]
            \item[Thank the critic.]
            \item[Respond to the style.]
            \item[Ask for time.]
        \end{description}

     \subsection*{Additional items from Messages.}
 
         \begin{description}
             \item[Acknowledgement.] \emph{Yes. I was half an hour late for work today. My car battery died.}
             \item[Clouding.] Use for statements that have a grain of truth but are intended mostly as insults.
                 \begin{itemize}
                     \item Agree in part. \emph{Yes. I do work more than 40 hours a week.}
                     \item Agree in probability. \emph{It could be that I work too much.}
                     \item Agree in principle. \emph{You're right. If I work too much, I will burn out.}
                 \end{itemize}
             \item[Probing.] \emph{What is it that bothers you about [important part of feedback]?}
         \end{description}
 
	\section*{Giving Corrective Feedback}

        \begin{description}
            \item[Choose your timing.]
            \item[Watch the ratio.]
            \item[Think before talking.]
            \item[Talk one-to-one.]
            \item[Frame the issue.]
            \item[Be precise.]
            \item[Include the positive in the message.]
            \item[Give information, not advice.]
            \item[Don't emote.]
        \end{description}

\end{multicols}

\begin{itemize}
    \item Paterson, R. J. (2000). The assertiveness workbook: How to express your ideas and stand up for yourself at work and in relationships. Oakland, CA: New Harbinger Publications.
    \item McKay, M., Davis, M., \& Fanning, P. (2009). Messages: The Communication Skills Book. Oakland, CA: New Harbinger Publications.
\end{itemize}


\end{document}
